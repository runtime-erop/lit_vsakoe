\documentclass{article}
\usepackage{graphicx}
\usepackage[russian]{babel}
\usepackage{hyperref}

\title{Мини-конспект по теме: Теорема Пифагора}
\author{Сыровацкий Егор}
\date{19 сентября 2025 г.}

\begin{document}

\maketitle

\tableofcontents

\newpage

\section{Введение}
Теорема Пифагора — одна из важнейших теорем евклидовой геометрии. Она находит применение в самых разных областях:

\begin{itemize}
    \item геомерия и тригонометрия
    \item физика
    \item инженерные расчёты
    \item компьютерная графика
\end{itemize}

\section{Формулировка теоремы}
\textbf{Слова:} В прямоугольном треугольнике квадрат гипотенузы равен сумме квадратов катетов.
\begin{center} $ c^2 = a^2 + b^2 $ \end{center} \begin{end} (1) \end{end}

\begin{center}
Как видно из формулы 1, знание двух сторон позволяет найти третью.
\end{center}

\section{Доказательство}
Одно из доказательств основывается на площади квадрата, составленного
из четырёх одинаковых прямоугольных треугольников и малого квадрата
в центре. Раскладывая площадь двумя способами, получаем $ c^2 = a^2 + b^2 $.

\section{Примеры расчёта}
\textbf{Пример 1}\\
\begin{center}
a = 3, b = 4\\
c = $\sqrt{3^2 + 4^2} = \sqrt{9 + 16} = \sqrt{25} = 5$
\end{center}
\textbf{Пример 2}\\
\begin{enumerate}
\item Дано: $ a=5,b=12 $
\item Решение:
\end{enumerate}
\begin{center}
a = 6, b = 8\\
c = $ \sqrt{6^2+8^2} = \sqrt{36+64} = \sqrt{100} = 10 $
\end{center}

\section{Таблица значений}
\begin{center}
\begin{tabular}{|c|c|c|}
\hline
Катет a & Катет b & Гипотенуза c \\
\hline
3 & 4 & 5 \\
\hline
5 & 12 & 13 \\
\hline
7 & 24 & 25 \\
\hline
\end{tabular}
\end{center}

\section{Иллюстрация}
\begin{center}
\includegraphics[width=0.5\textwidth]{triangle.png}
\end{center}

\section{Заключение}
Теорема Пифагора — один из краеугольных камней геометрии, помогающий решать
множество практических задач

\section{Ссылки и литература}
\begin{itemize}
    \item Википедия: \href{https://ru.wikipedia.org/wiki/%D0%A2%D0%B5%D0%BE%D1%80%D0%B5%D0%BC%D0%B0_%D0%9F%D0%B8%D1%84%D0%B0%D0%B3%D0%BE%D1%80%D0%B0}{Теорема Пифагора}
    \item Классические учебники геометрии
\end{itemize}

\end{document}